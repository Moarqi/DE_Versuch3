Der ZVN3306A ist ein N-Kanal-Transistor, der ZVP3306A ein P-Kanal Transistor. Daraus ergibt sich folgender wichtier Unterschied: der N-Kanal-Transistor schaltet bei bei einer positiven Gate-Source-Spannung (sofern diese die Schwellspannung überwindet), der P-Kanal-Transistor hingegen schaltet nur bei einer negativen Gate-Source-Spannung[REFERENZ SKRIPT Seite 33]. Dies ist bedingt durch den inneren Aufbau:

p-Kanal MOSFETs haben als Halbleiter zwischen Drain und Source ein n-dotiertes Metall, welches im Kristall-Gitter an manchen Stellen fünf-wertige Atome an Stelle der vier-wertigen Silicium-Atome besitzt. Dadurch gibt es an diesen Stellen einen Elektronenüberschuss, dieses zusätzliche Atom ist frei beweglich. Wird jetzt am Bulk eine positive Spannung angelegt, so wird das Elektron zum Bulk hingezogen und und es entsteht ein Bereich mit positivem Potenzial am fünf-wertigen Atom. Dieses zieht andere, frei bewegliche Elektronen an und ermöglicht damit eine Ladungsträger Bewegung zwischen Drain und Source durch den Halbleiter.

Für n-Kanal MOSFETS sieht das ganze etwas anders aus. Das Halbleiter Material ist p-dotiert, es wurden also fünf-wertige Atome in das vier-wertige Silicium gebracht. Der Effekt ist dadurch genau umgekehrt: es gibt Elektronen-Lücken, die von freien Elektronen gefüllt werden können. Damit aber eine solche Lücke gefüllt wird, muss eine andere freigegeben werden. Dies lässt sich dann auch vereinfacht als bewegliche positive Ladungsträger betrachten. Legen wir nun eine negative am Bulk an, so werden diese positiven Ladungsträger "aus dem Weg gezogen" und ein Stromfluss zwischen Drain und Source durch den Ladungsträger ist möglich.